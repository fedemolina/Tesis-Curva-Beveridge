\documentclass[msc,oneside,a4paper]{udelar}\usepackage[]{graphicx}\usepackage[]{color}
% maxwidth is the original width if it is less than linewidth
% otherwise use linewidth (to make sure the graphics do not exceed the margin)
\makeatletter
\def\maxwidth{ %
  \ifdim\Gin@nat@width>\linewidth
    \linewidth
  \else
    \Gin@nat@width
  \fi
}
\makeatother

\definecolor{fgcolor}{rgb}{0.345, 0.345, 0.345}
\newcommand{\hlnum}[1]{\textcolor[rgb]{0.686,0.059,0.569}{#1}}%
\newcommand{\hlstr}[1]{\textcolor[rgb]{0.192,0.494,0.8}{#1}}%
\newcommand{\hlcom}[1]{\textcolor[rgb]{0.678,0.584,0.686}{\textit{#1}}}%
\newcommand{\hlopt}[1]{\textcolor[rgb]{0,0,0}{#1}}%
\newcommand{\hlstd}[1]{\textcolor[rgb]{0.345,0.345,0.345}{#1}}%
\newcommand{\hlkwa}[1]{\textcolor[rgb]{0.161,0.373,0.58}{\textbf{#1}}}%
\newcommand{\hlkwb}[1]{\textcolor[rgb]{0.69,0.353,0.396}{#1}}%
\newcommand{\hlkwc}[1]{\textcolor[rgb]{0.333,0.667,0.333}{#1}}%
\newcommand{\hlkwd}[1]{\textcolor[rgb]{0.737,0.353,0.396}{\textbf{#1}}}%
\let\hlipl\hlkwb

\usepackage{framed}
\makeatletter
\newenvironment{kframe}{%
 \def\at@end@of@kframe{}%
 \ifinner\ifhmode%
  \def\at@end@of@kframe{\end{minipage}}%
  \begin{minipage}{\columnwidth}%
 \fi\fi%
 \def\FrameCommand##1{\hskip\@totalleftmargin \hskip-\fboxsep
 \colorbox{shadecolor}{##1}\hskip-\fboxsep
     % There is no \\@totalrightmargin, so:
     \hskip-\linewidth \hskip-\@totalleftmargin \hskip\columnwidth}%
 \MakeFramed {\advance\hsize-\width
   \@totalleftmargin\z@ \linewidth\hsize
   \@setminipage}}%
 {\par\unskip\endMakeFramed%
 \at@end@of@kframe}
\makeatother

\definecolor{shadecolor}{rgb}{.97, .97, .97}
\definecolor{messagecolor}{rgb}{0, 0, 0}
\definecolor{warningcolor}{rgb}{1, 0, 1}
\definecolor{errorcolor}{rgb}{1, 0, 0}
\newenvironment{knitrout}{}{} % an empty environment to be redefined in TeX

\usepackage{alltt}

\usepackage[
acronyms, 											 %utiliza el glosario de acronimos.
nohypertypes={acronym,notacion,simbolos,glosario},   %quita los links en el texto al glosario.
%nonumberlist,                                       %quita los links en los glosarios al texto.
nogroupskip,                                         %quita los espacios entre diferentes grupos dentro de un glosario.
nopostdot, 											 %quita el punto final en los acrónimos         .                           
]{glossaries}

\hypersetup{ colorlinks = true } % Hipervínculos: escribir "false" para imprimir o "true" para ver en digital.

% Ver los documentos de estilos bibliográficos para editar estas siguientes 2 líneas. Se deben de copiar a partir de los PDF de estilos bibliográficos y no es necesario que el estudiante las edite.
\usepackage{natbib} % Para algunos estilos bibliográficos
\bibliographystyle{estilos_bibliograficos/natbib/apalike}

\loadglossary


% \bibliography{bibliografia/TesisBeveridgeCurve.bib}
\IfFileExists{upquote.sty}{\usepackage{upquote}}{}
\begin{document}
\SweaveOpts{concordance=TRUE, echo=FALSE, message=FALSE, warning=FALSE, cache=FALSE}
    
    \title{Título del proyecto}
    \subtitle{Subtítutlo}
    \institutelogo{1} % Carga cantidad de logos seleccionados, con máximo de 3 logos.
    \author{Nombres integrantes del proyecto}{Apellidos integrantes del proyecto}
    \escritura{de} % Se indica que el programa de Posgrado sea "en" o "de" tal área.
    %
    \director{Prof.}{Nombre del Director de Tesis}{Apellido}{D.Sc.}
    \director{Prof.}{Nombre del Director de Tesis}{Apellido}{D.Sc.} % Comentar esta línea si se tiene solo un director de tesis.
    \codirector{Prof.}{Nombre del 1er Codirector}{Apellido}{D.Sc.}  % Comentar estas líneas si no son necesarias.
    \codirector{Prof.}{Nombre del 2do Codirector}{Apellido}{D.Sc.}
    \codirector{Prof.}{Nombre del 3er Codirector}{Apellido}{D.Sc.}
    \directoracademico{Prof.}{Nombre del Director Académico de Tesis}{Apellido}{D.Sc.}
    %
    \examiner{Prof.}{Nombre del 1er Examinador}{Apellido}{D.Sc.}
    \examiner{Prof.}{Nombre del 2do Examinador}{Apellido}{Ph.D.}
    \examiner{Prof.}{Nombre del 3er Examinador}{Apellido}{D.Sc.}
    \examiner{Prof.}{Nombre del 4to Examinador}{Apellido}{Ph.D.}
    \examiner{Prof.}{Nombre del 5to Examinador}{Apellido}{Ph.D.} % Comentar los que no son necesarios.
    %
    \graduatename{Fondo Sectorial de Datos}
    \institute{Facultad de Economía, Instituto de Estadística}{FCEA-IESTA}  % La primer institución es la principal.
    % \institute{Facultad de Medicina}{FMed}  % Agregar copias de esta línea para agregar instituciones.
    % \institute{Facultad de Agronomía}{FAgro}  % Agregar copias de esta línea para agregar instituciones. No se discrimina de que universidad es tal facultad.
    % \seconduniversity{Universidad XXXXX} % Se agrega el nombre de otra universidad. Comentar esta linea si no es necesaria.
    % \thirduniversity{Universidad YYYYY} % Se agrega el nombre de otra universidad. Comentar esta linea si no es necesaria.
    \graduatelocation{Montevideo}{Uruguay}
    %
    \date{31}{12}{2019} % Fecha del documento: día/mes/año
    % Palabras claves en español
    \keyword{1ra palabra clave}
    \keyword{2da palabra clave}
    \keyword{3ra palabra clave}
    \keyword{4ta palabra clave}
    \keyword{5ta palabra clave}
    % Palabras claves en inglés
    \foreignkeyword{1st keyword}
    \foreignkeyword{2nd keyword}
    \foreignkeyword{3rd keyword}
    \foreignkeyword{4th keyword}
    \foreignkeyword{5th keyword}
    % %     maketitle & frontmatter no funcionan.
    \maketitle
    % Comando que genera el título de la tesis.
    % \frontmatter
    % Comando que genera la portadilla, el catalogo y el tribunal de evaluación. NO COMENTAR
    % \SweaveInput{ded_agr_epi/dedicacion.Rnw}
    \SweaveInput{ded_agr_epi/agradecimientos.Rnw}
    % \SweaveInput{ded_agr_epi/epigrafe.Rnw}
    \SweaveInput{resumen/resumen.Rnw}
    \SweaveInput{resumen/abstract.Rnw}
    
    % \listadesimbolos
    % Lista de símbolos
    % \listadesiglas
    % Lista de siglas
    \tableofcontents
    % Tabla de contenidos. Compilar dos veces para ver los cambios completos.
    \mainmatter
    % Comando que genera las listas y capítulos. NO COMENTAR
    % Se incluyen los capítulos. Se pueden comentar los capítlos en los cuales no se está trabajando, para que el documento de trabajo sea más pequeño y compile más rápido.
    \SweaveInput{capitulos/chap01.Rnw}
    \SweaveInput{capitulos/chap02.Rnw}
    \SweaveInput{capitulos/chap03.Rnw}
    \SweaveInput{capitulos/chap04.Rnw}
    \SweaveInput{capitulos/chap05.Rnw}
    % % Seguir copiando la linea de arriba para agregar más capítulos.
    % \input{PROOF/Child.Rnw}

% 
  \backmatter 
  % Comando que generalos apéndices, anexos y bibliografía. NO COMENTAR
  \bibliography{bibliografia/TesisBeveridgeCurve.bib}
  % Agregar la cantidad de archivos .bib que se tengan para la bibliografía.
  \bibend
  % No comentar
  \glosario
  % Glosario, NO comentar
  \apenarabicnumbering
  \apenmatter
  % Apéndices, NO comentar
  \SweaveInput{apendice/apendice_A.Rnw}
  \SweaveInput{apendice/apendice_B.Rnw}
%   % Seguir copiando la linea de arriba para agregar más apéndices.
%   %
  \anexarabicnumbering
  \anexmatter
  % Anexos, NO comentar
 \SweaveInput{anexo/anexo_A.Rnw}
  % Seguir copiando la linea de arriba para agregar más anexos.
  % 

\end{document}
