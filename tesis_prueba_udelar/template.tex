% ===== INICIO DEL PREÁMBULO =====
\documentclass[msc,oneside,a4paper]{udelar} % Poner msc para Maestría, dsc para Doctorado.
\usepackage[
acronyms, 											 %utiliza el glosario de acronimos.
nohypertypes={acronym,notacion,simbolos,glosario},   %quita los links en el texto al glosario.
%nonumberlist,                                       %quita los links en los glosarios al texto.
nogroupskip,                                         %quita los espacios entre diferentes grupos dentro de un glosario.
nopostdot, 											 %quita el punto final en los acrónimos         .                           
]{glossaries}
\hypersetup{colorlinks = true} % Hipervínculos: escribir "false" para imprimir o "true" para ver en digital.
% Ver los documentos de estilos bibliográficos para editar estas siguientes 2 líneas. Se deben de copiar a partir de los PDF de estilos bibliográficos y no es necesario que el estudiante las edite.
\usepackage{natbib} % Para algunos estilos bibliográficos
\bibliographystyle{estilos_bibliograficos/natbib/apalike}
\loadglossary % No comentar esta línea


%%%%%%%%% Agregar lo que va en el yaml
% To pass between YAML and LaTeX the dollar signs are added by CII
\title{$title$}
\subtitle{$subtitle$}
\institutelogo{$institutelogo$}
\author{$author$}
\escritura{$escritura$}
\director{$director$}
\codirector{$codirector$}
\directoracademico{$directoracademico$}
\examiner{$examiner$}
\graduatename{$graduatename$}
\institute{$institute$}
\seconduniversity{$seconduniversity$}
\thirduniversity{$thirduniversity$}
\graduatelocation{$graduatelocation$}
\date{$date$}
\keyword{$keyword$}
\foreignkeyword{$foreignkeyword$}

% Added by CII
%%% Copied from knitr
%% maxwidth is the original width if it's less than linewidth
%% otherwise use linewidth (to make sure the graphics do not exceed the margin)
% \makeatletter
% \def\maxwidth{ %
%   \ifdim\Gin@nat@width>\linewidth
%     \linewidth
%   \else
%     \Gin@nat@width
%   \fi
% }
% \makeatother
% \renewcommand{\contentsname}{Table of Contents}
% End of CII addition
% \setlength{\parskip}{0pt}
% Added by CII
% $if(space_between_paragraphs)$
%   %\setlength{\parskip}{\baselineskip}
%   \usepackage[parfill]{parskip}
% $endif$
% \providecommand{\tightlist}{%
%   \setlength{\itemsep}{0pt}\setlength{\parskip}{0pt}}
  
  
\Agradecimientos{
$agradecimientos$
}
\Dedicacion{
$dedicacion$
}
\Preface{
$preface$
}
\Abstract{
$abstract$
}
$for(header-includes)$
	$header-includes$
$endfor$

%%%%%%%%%%%%%%%%%%%%%%%%%%%%

\begin{document}
  %
$if(title)$
  \maketitle
$endif$
$if(subtitle)$
  \begin{subtitle}
    $subtitle$
  \end{subtitle}
$endif$
$if(institutelogo)$
  \institutelogo{3}
$endif$
$if(author)$
  \begin{author}
    $author$
  \end{author}
$endif$
$if(escritura)$
  \begin{escritura}
    $escritura$
  \end{escritura}
$endif$
$if(director)$
  \begin{director}
    $director$
  \end{director}
$endif$
$if(codirector)$
  \begin{codirector}
    $codirector$
  \end{codirector}
$endif$
$if(directoracademico)$
  \begin{directoracademico}
    $directoracademico$
  \end{directoracademico}
$endif$
$if(examiner)$
  \begin{examiner}
    $examiner$
  \end{examiner}
$endif$
$if(graduatename)$
  \begin{graduatename}
    $graduatename$
  \end{graduatename}
$endif$
$if(institute)$
  \begin{institute}
    $institute$
  \end{institute}
$endif$
$if(seconduniversity)$
  \begin{institute}
    $seconduniversity$
  \end{seconduniversity}
$endif$
$if(thirduniversity)$
  \begin{thirduniversity}
    $thirduniversity$
  \end{thirduniversity}
$endif$
$if(graduatelocation)$
  \begin{graduatelocation}
    $graduatelocation$
  \end{graduatelocation}
$endif$
$if(graduatelocation)$
  \begin{graduatelocation}
    $graduatelocation$
  \end{graduatelocation}
$endif$
% \date{$date$}
$if(keyword)$
  \begin{keyword}
    $keyword$
  \end{keyword}
$endif$
$if(foreignkeyword)$
  \begin{foreignkeyword}
    $foreignkeyword$
  \end{foreignkeyword}
$endif$

  %
  \frontmatter  % Comando que genera la portadilla, el catalogo y el tribunal de evaluación. NO COMENTAR
  %
  % \include{ded_agr_epi/dedicacion}
  % \include{ded_agr_epi/agradecimientos}
  % \include{ded_agr_epi/epigrafe}
  % \include{resumen/resumen}
  % \include{resumen/abstract}
  %
  \listoffigures	         % Lista de figuras
  \listoftables	         % Lista de tablas
  \listadesimbolos 		     % Lista de símbolos
  \listadenotaciones 	     % Lista de notaciones
  \listadesiglas 		     % Lista de siglas
  %
  \tableofcontents           % Tabla de contenidos. Compilar dos veces para ver los cambios completos.
  %
  \mainmatter % Comando que genera las listas y capítulos. NO COMENTAR
  %
  % Se incluyen los capítulos. Se pueden comentar los capítlos en los cuales no se está trabajando, para que el documento de trabajo sea más pequeño y compile más rápido.
  % \chapter{Introducción}

Este trabajo se plantea 3 objetivos: 1) Generar una serie de vacantes laborales de frecuencia trimestral desde 1980 hasta 2018. 2) En base a los datos obtenidos estimar una curva de Beveridge para el mismo periodo. 3) Responder si existe algún corrimiento, cambio de pendiente y/o movimiento sobre la curva de Beveridge. El análisis en los tres casos se restringe al departamento de Montevideo, Uruguay.

Como sugieren \cite{Bergara2017} las transformaciones sufridas por la economía uruguaya en los 15 años han sido de carácter estructural, al igual que lo fueron las implementadas luego de la dictadura y hasta el año 2000 analizadas por \cite{Antia2001}. Por tanto, la hipótesis es que al menos uno de los 3 sucesos respecto a la curva de Beveridge se materializó. Para responder la pregunta se construye una serie de vacantes laborales para el periodo de análisis 1980-2018 en base \cite{Rama1988}, \cite{Urrestarazu1997}, \cite{Ceres2012} y datos propios, siendo siempre la fuente principal el Diario El País, clasificados laborales, ``Gallito''. Se utiliza como estrategia empírica vectores bayesianos autorregresivos con parámetros variables y volatilidad estocástica (TVP-VAR) siguiendo a \cite{Nakajima2011, Benati2013, Primiceri2005, Lubik2016b}, por dos motivos básicos 1) permite levantar el supuesto de una relación invariante en la relación vacantes-desempleo y 2) modelizar de forma no lineal. El marco de análisis para identificar shocks estructurales es la teoría de búsqueda y emparejamiento resumida en \cite{Pissarides2000}, siguiendo a \cite{Benati2013} quienes construyen sobre \cite{Shimer2005}.

\cite{Blanchard1989} remarcan que los pensamientos de los macroeconomistas respecto a la dinámica agregada del mercado laboral han sido organizados en base dos relaciones, la curva de Phillips y la curva de Beveridge. Sin embargo, la curva de Beveridge, entendida como la relación entre vacantes laborales y desempleo, ha jugado un rol notoriamente secundario. El caso uruguayo no es ajeno al no existir trabajos académicos al respecto en los últimos 20 años.%, de allí surge una primera motivación. Buscar llenar un vacío de la economía laboral uruguaya.
 
La curva de Beveridge o curva UV plantea una relación gráfica convexa hacia el origen entre dos variables, vacantes laborales y desempleo\footnote{Ambas expresadas como tasas con respecto a una tercera variable Población Económicamente Activa (PEA), Población en Edad de Trabajar (PET) o trabajadores totales (L). Gráficamente las vacantes suelen estar en las ordenadas y la tasa de desempleo en las abscisas}. El desempleo se entiende como personas que buscan trabajo remunerado activamente pero no logran obtenerlo. Mientras las vacantes laborales, como aquella posición dentro de la firma que el empleador busca llenar activamente en un periodo a determinar, lo cual suele usarse como proxy de la demanda laboral.
Mientras el desempleo suele ser una variable cuya definición y calculo esta generalizado, las vacantes son raramente calculadas bajo una metodología sistemática, no suelen ser recolectadas, ni existen encuestas al respecto a excepción de algunos pocos países de la OCDE\footnote{Para una lista de países con índices de vacantes laborales y que se ha trabajo sobre la curva de Beveridge, ver los trabajores de \cite{Hobijn2013, Nickell2002}}.

Su nacimiento se debe principalmente a los trabajos seminales de \cite{Beveridge} y \cite{Dicks-Mireaux1958}, estos últimos quienes plantean la relación gráfica por primera vez y, remarcan la robustez del indicador de vacantes laborales en cuanto a su uso de forma cualitativa, permitiendo fuera ampliamente utilizada durante los años 1960-1970. \cite{Rodenburg2007} analiza la evolución de la curva de Beveridge y muestra como ocupó un rol secundario en la macroeconomía entre 1970 y 1980, lo cual entiende pudo deberse entre otros factores, a su ausencia de microfundamentos y al asentamiento de la escuela neoclásica como mainstream. Sin embargo, con el nacimiento de la teoría de búsqueda y emparejamiento bajo los trabajos seminales de \cite{Pissarides1985}, \cite{Mortensen1994} y \cite{Diamond1982} conocido popularmente como el marco de análisis DMP, resumido en \cite{Pissarides2000}, y especialmente el trabajo de \cite{Blanchard1989} la curva de Beveridge recobró importancia en el análisis macroeconómico.\footnote{Para ver la evolución hasta 1986, ver \cite{Mortensen1986}. Para una investigación actual ver \cite{Elsby2015}}  

Respecto a su relevancia \cite{Blanchard1989} plantean que la curva de Beveridge conceptualmente viene primero que la curva de Phillips, y contiene información esencial sobre el funcionamiento del mercado de trabajo y los shocks que afectan al mismo. \cite{Elsby2015} muestran que se utiliza a nivel macroeconómico como un marco de análisis para el entendimiento de los mercados laborales tanto a nivel agregado como desagregado, para el análisis de la volatilidad y coexistencia de desempleo y vacantes, a la vez que como una aproximación al proceso de matching del mercado laboral.

El proceso de matching es el siguiente, un trabajador activo busca un trabajo que cumpla sus requisitos, para ello incurre en costos de búsqueda, gasta tiempo, dinero y deja de realizar actividades con beneficios potenciales. Una vez encontrado el puesto deseado, puede obtenerlo o no. Ello se debe a que el trabajador no solamente valora el ingreso salarial y las condiciones laborales presentes, lo cual es una decisión estática, sino que toma en cuenta al menos dos factores extras, el flujo de ingresos futuros y las condiciones laborales futuras (factores no monetarios).

La firma con vacantes disponibles, busca un trabajador acorde al puesto, en su búsqueda incurre en costos. Una vez encontrado el trabajador deseado, puede ocurrir que no se efectivice su contratación debido a que la firma al momento de ofrecer el contrato laboral debe internalizar no solamente los costos de búsqueda sino también los costos de capacitación. Además debe tomar en cuenta que la relación contractual puede finalizar porque el empleado decide renunciar o bien sea la firma quién finaliza la misma por shocks adversos o porque el candidato elegido quien parecía el indicado, realmente no lo era.

Esto muestra que el matching, en el cual se encuentra la oferta con la demanda laboral son procesos de decisión inherentemente dinámicos y descoordinados por parte del trabajador y la firma que conllevan fricciones, riesgos, asimetrías de información e incertidumbre, generando una diferencia sistemática entre demanda y oferta laboral más allá del desempleo voluntario.

El proceso de matching se relaciona con los diferentes componentes del desempleo: voluntario, desequilibrio o estructural y segmentación o friccional analizados en \cite{Rama1988}. 
El análisis de la curva de Beveridge mediante la teoría de búsqueda y emparejamiento puede explicar los 3 componentes, al menos de forma cualitativa\footnote{Es importante remarcar que la cuantificación que se haga de los efectos siempre va a estar sujeta a la calidad de la serie de vacantes, la cual suele estar sujeta a errores de medición y este trabajo no es la excepción. Sin embargo, su capacidad de análisis cualitativa suele ser confiable, ver \cite{Dicks-Mireaux1958} a la vez que existen posibles mejoras, ver \cite{Abraham1987}}.

Respecto al desempleo voluntario, es decir, trabajadores dispuestos a trabajar que no trabajan porque consideran el salario ofrecido insuficiente. Puede verse mediante cambios en la PEA, shocks que generen ingresos o egresos de personas dispuestas a trabajar provocan corrimiento hacia el origen (caída de la PEA) o hacia afuera (aumento de la PEA). 

Cambios en el desempleo friccional, falta de correspondencia entre los puestos y candidatos, en especial explicado por factores institucionales que logren afectar el proceso de contratación como los costos de contratación y despido, el salario mínimo, el derecho a huelga y ocupación, movilidad de la mano de obra, menores tasas de sindicalización, mejora en la tecnología del matching debido al uso de la tecnología como avisos web, entre otros. Se puede analizar como cambios en la función de matching, implicando un posible mejoramiento o empeoramiento de la eficiencia del mercado laboral. Un corrimiento hacia afuera de la curva implicaría un empeoramiento del match, mientras un corrimiento hacia dentro una mejora.

Finalmente, el componente estructural, la escasez (exceso) global de puestos de trabajo con respecto a los candidatos disponibles, o sea, un exceso (insuficiencia) de oferta laboral. Esto último debido a la relación inversa entre vacantes y desempleo, cuando el desempleo es elevado las vacantes son escasas y viceversa.

El punto en cual nos encontramos sobre la curva puede estar indicando en que parte del ciclo económico se sitúa la economía. Si la posición es un punto con altas vacantes y bajo desempleo, la economía estaría en una etapa expansiva, parte alta del ciclo económico. Contrariamente, un punto con bajas vacantes y elevado desempleo indicaría que la economía esta en una fase recesiva, parte baja del ciclo económico.

Pese a su trascendencia, su análisis en Uruguay esta desactualizado, ya que, el último trabajo al respecto es de \cite{Urrestarazu1997}. Una explicación puede venir por la ausencia de una serie de vacantes pública, sistemática y regular. Sin embargo, entiendo no puede ser la única explicación por cuatro motivos.
 
Primero, el Ministerio de Trabajo y Seguridad Social (MTSS), recolectó información sobre vacantes laborales entre década de 1970 y 1980. Siendo descontinuado por motivos desconocidos. Recién a partir de 2016 el MTSS en conjunto con diario El País publican un informe sobre el marcado laboral en base a las vacantes publicadas en el diario El País, sección ``Gallito Luis''. A su vez el Instituto Nacional de Estadística (INE) que tiene por objetivo la elaboración, supervisión y coordinación de las estadísticas nacionales nunca ha recolectado información ni ha generado encuestas sobre vacantes laborales. Segundo, la existencia del Índice Ceres de Demanda Laboral (ICDL) de frecuencia mensual para 1998-2014, que si bien no es público, se publicaba regularmente. Tercero, por la facilidad y total disposición del diario El País a que estudiantes, investigadores u organismos públicos utilicen su información con fines académicos. Esto queda de manifiesto dado que Urrestarazu accede a dicha información en 1996, yo accedo a la misma en 2018 y el MTSS desde 2016. Cuarto, todas las publicaciones de El País, sección laboral, ``El Gallito'' se encuentran disponibles en la biblioteca Nacional\footnote{En este último punto vale remarcar el enorme trabajo realizado por \cite{Alma2011}, quienes recolectan datos para distintos meses entre 2000 a 2010 permitiendo analizar potenciales sesgos del ``Gallito'' en cuanto puestos de baja calificación, tanto a nivel agregado como por industrias}.

%. De hecho, el Instituto Nacional de Estadística (INE) en Uruguay "que tiene por objetivo la elaboración, supervisión y coordinación de las estadísticas nacional"\footnote{link} nunca ha relevado datos de avisos laborales ni ha realizado encuestas al respecto. El único organismo público que ha realizado un relevamiento al respecto fue el Ministerio de Trabajo y Seguridad Social (MTSS) sobre finales de 1970 y mediados de 1980 para posteriormente retomarlo a partir del año 2016 utilizando como única fuente de información los avisos laborales publicados en el diario El País, sección Gallito. Si bien han existido iniciativas desde el ámbito académico\cite{Alma2011} y privado\cite{Ceres2012} las mismas han quedado descontinuadas y han sido insuficientes, en la medida que no han realizado un análisis del comportamiento de la CB.

Por ello, este trabajo tiene como uno de sus objetivos sistematizar los datos existentes respecto a vacantes laborales recabados en trabajos previos cuya metodología permite una armonización de las series de vacantes. A continuación extender el periodo de análisis en base a información recabada mayoritariamente a partir del diario El País, clasificados laborales, ``Gallito'' con el fin de construir un índice de vacantes laborales, en adelante ``Índice de Vacantes Laborales'' (IVL). De esta forma se da un paso importante en la generación de un índice de vacantes laborales de carácter público, prolongado y que permita ser perfeccionado posteriormente dotando de una herramienta de análisis macroeconómico de la cual Uruguay actualmente adolece.

%Finalizada la creación del IVL, se procede a su análisis estadístico tanto de forma individual como en conjunto con la tasa de desempleo. Esto es relevante, ya que, al no existir previamente para nuestro período de análisis no conocemos sus propiedades estadísticas ni si existe una relación conjunta entre el IVL y la tasa de desempleo, se espera que las mismas estén cointegradas.

La pregunta de investigación, se fundamenta por 3 aspectos: 1) los trabajos previos para Uruguay, en donde se gráfica y estima la curva de Beveridge por \cite{Rama1988, Urrestarazu1997, DECON1993} en los cuales no se rechaza la hipótesis de quiebres estructurales (tanto de nivel como pendiente). 2) el análisis de las series de vacantes existentes y su elevada correlación con el PIB. 3) La literatura, marco DMP, siguiendo \cite{Rodenburg2007} y \cite{Elsby2015} que plantea que shocks o cambios estructurales en la economía generan corrimientos de la curva (hacia afuera o adentro, debido a un aumento o disminución del 'mismatch'), al igual que políticas públicas que busquen aumentar la efectividad del match en el mercado laboral volviéndolo más eficiente (movimiento hacia adentro) y que los movimientos sobre la curva se asocian al ciclo económico. 

Siguiendo a \cite{Antia2001} Uruguay en la década de los 90 transito una apertura de su cuenta corriente y de capitales que genero grandes transformaciones en el mercado laboral, posteriormente tuvo una grave crisis económica entre 2001-2002 y luego un crecimiento promedio anual del PIB entre 2003 y 2009 mayor al 5\% junto a una tasa de desempleo en torno a 5-6\%. A continuación, \cite{Bergara2017} muestra que se generaron un conjunto de reformas estructurales a partir de 2005, en el sector salud, tributario, financiero y social. Finalmente la economía transita desde 2010-2011 un enlentecimiento de la actividad, agravado a partir de 2014-2015, situándose actual y técnicamente en recesión, en conjunto con una tasa de desempleo en torno al 8-9\% y una transformación del mercado laboral en el cual aumenta la sustitución de mano de obra por capital y nuevos procedimientos basados en innovaciones tecnológicas. Por ello, es razonable plantear la hipótesis de que la curva de Beveridge tuvo al menos un corrimiento de nivel, pendiente o un movimiento sobre si misma entre 1980 y 2018. 

%Por lo tanto, surge la siguiente interrogante ¿Existe algún corrimiento y/o cambio de pendiente en la Curva de Beveridge para Uruguay entre los años 1980 y 2018? Dicha pregunta es el objetivo principal de este trabajo de investigación y una de sus motivaciones principales. Como se adelantó, la hipótesis del trabajo es que debe existir al menos un quiebre o cambio de pendiente en la CB para el periodo de análisis. 

Posteriormente se plantea la estimación de la curva de Beveridge mediante una estrategia empírica de TVP-SVAR para analizar cambios de la relación entre vacantes y desempleo, identificado bajo un modelo básico de búsqueda y emparejamiento. Sin embargo, para poder llegar a esta especificación primero debe ponerse a prueba la hipótesis de parámetros variables a lo largo del tiempo y volatilidad estocástica, los cuales siguen un proceso de markov de orden uno. Para ello, se sigue a \cite{Benati2013} quienes utilizan los test desarrollados por \cite{Stock1996, Stock1998} para testear la presencia de un camino aleatorio entre vacantes y desempleo.

El trabajo es organizado de la siguiente forma. La sección siguiente revisa la literatura de la curva de Beveridge en Uruguay y otros países del mundo. Sección 3 define las variables y fuentes de datos a utilizar, a la vez que detalla la metodología utilizada para armonizar las series de vacantes. Sección 4 especifica el marco teórico utilizado para obtener las restricciones de identificación. Sección 5 define la estrategia empírica elegida, discutiendo . Sección 6 especifica los resultados del trabajo. Sección 7 discute al respecto de estos últimos. Sección 8 plantea algunas conclusiones.
 % Se carga el capítulo 01
  % \chapter{Fundamentos teóricos}

Según \cite{Rodenburg2007} el uso de la curva de Beveridge ha pasado por periodos de amplia y baja utilización. Pese a ello, \cite{Elsby2015} muestra que ha logrado volverse un marco de análisis central en la macroeconomía, en especial la economía laboral. Ello se ha logrado fundamentalmente por los trabajos de \cite{Pissarides1985}, \cite{Mortensen1994} y \cite{Diamond1982} quienes dieron microfundamentos a la relación entre vacantes y desempleo de forma de alinear a la misma dentro de la economía neoclásica y, en especial al trabajo de \cite{Blanchard1989} que colocó la curva de Beveridge en la primera escena de la macroeonomía. La importancia de haber dado microfundamentos radica en que la misma pasa de ser un herramental descriptivo (como ha sido utilizada en Uruguay), a uno  de carácter explicativo e interpretativo de extrema utilidad para la política económica. Esto último no quiere decir que no halla sido utilizada previamente con dichos fines. 

Su aplicación en Uruguay puede remontarse a \cite{Rama1988}, \cite{DECON1993}, \cite{Urrestarazu1997} y \cite{Alma2011}, mientras que para un análisis de distintos países de la OCDE hasta 1990 puede verse \cite{Nickell2002} y hasta 2013 \cite{Hobijn2013}. En el caso de \cite{Rama1988}, su utilización fue con fines descriptivos para aproximarse a la descomposición de la desocupación en desempleo voluntario, de segmentación y desequilibrio. Su trabajo es el primero en Uruguay en plantear gráficamente la curva de Beveridge (1978-1988) y es quien construye el primer índice de vacantes conocido. Su objetivo se centra en cuantificar los componentes de la desocupación analizando la agregación de micro mercados de trabajo mediante un modelo de desequilibrio. Plantea que la curva de Beveridge puede utilizarse para cuantificar los componentes de segmenación y desequilibrio, no así el voluntario, debido a las variaciones de la PEA en la década de 1980.

\cite{Urrestarazu1997} construye sobre Rama, extiende el periodo de análisis hasta el año 1995 y, si bien su foco es al igual que Rama el análisis del desempleo de segmentación y los micro mercados mediante modelos de desequilibrio, estima por mínimos cuadrados ordinarios (MCO) una curva de Beveridge. Mediante una estimación restringida y otra no restringida (test Chow) pone a prueba la hipótesis de que exista un cambio estructural en 1990. Sus resultados concluyen que no se rechaza la hipótesis de existencia de un cambio estructural en la curva de Beveridge. Sin embargo, advierte siguiendo planteos formulados por Rama que las conclusiones obtenidas solo son válidas en la medida que la evolución del desempleo voluntario sea estable, característica común en países desarrollados, pero no así en Uruguay. 

Por su parte, \cite{DECON1993} estiman mediante una regresión lineal una curva de Beveridge entre 1980 y 1990 en la cual encuentran evidencia de una tendencia de la curva a desplazarse hacia afuera, indicando un mercado de trabajo con mayor rigidez, aunque se demarcan que la razón fundamental de ello sean salarios relativos inadecuados. 

Por último, \cite{Alma2011}, la utilizan en sus propias palabras ``con fines ilustrativos'' para mostrar una relación inversa y negativa para el periodo 2000-2009. En la misma se observa, lo que parece indicar un cambio de pendiente en la curva de Beveridge, sin embargo, como los mismos autores remarcan dada la escasez de datos no es posible identificar alteraciones relevantes. % Se carga el capítulo 02
  % \chapter{Metodología}

%\section{Marco Teórico}

El marco teórico del proyecto es la curva de Beveridge, que plantea una relación convexa hacia el origen entre vacantes laborales y desempleo, como puede observarse en la figura 4 del apéndice. Por tanto, los indicadores de los conceptos claves sobre los que se trabaja son el índice de vacantes laborales y la tasa de desempleo. Sin embargo, es posible enmarcar la curva de Beveridge bajo la teoría de búsqueda y emparejamiento. De esta forma, la misma se deriva en el modelo básico \cite{Pissarides2000} a partir de suponer la existencia de una función de matching, de suponer un proceso estocástico de Poisson mediante el cual se llenan las vacantes y que en un estado estacionario la tasa de variación de la tasa de desempleo debe ser nula. La función de matching, es un artilugio similar a la función de producción, es una caja negra mediante la cual se genera el matching entre vacantes laborales y desempleados \cite{Pissarides2000}\footnote{En esta sección me abstraigo del desarrollo y relación de la curva de Beveridge con la curva de Job Creation}. Conceptualmente la idea es que, modificaciones en la función de matching generan corrimiento de la curva de Beveridge hacia el origen o hacia afuera. Así como modificaciones en la función de producción generan una economía con mayor o menor capacidad de producción, la función de matching refleja una economía donde el match entre trabajador y firma puede ser más rápido o lento, por lo tanto, el mercado laboral puede ser más o menos eficiente. En la ecuación \eqref{eq7} del apéndice puede observarse como modificaciones de $\theta$, que es el cociente entre vacantes y tasa de desempleo, genera movimientos sobre la curva, asociados al ciclo económico. Mientras modificaciones de q($\theta$), que reflejen cambios en la función de matching $m$ al igual que alteraciones en $\lambda$, que puede asociarse a la incertidumbre que sufre el trabajador de perder los beneficios de un puesto ocupado, van a generar corrimientos de la curva.

Modificaciones en la función de matching, pueden asociarse a una diferencia entre las habilidades requeridas por las firmas y las ofrecidas por los trabajadores. Esto puede darse si los cambios tecnológicos son sesgados hacia la utilización del capital, nueva tecnología y personas con alta capacitación, siendo estás últimas difíciles de encontrar. Si este fuese el caso, tanto las vacantes laborales como la tasa de desempleo pueden aumentar, o bien aumentar solamente el desempleo para una misma tasa de vacantes. Los shocks sobre la función de matching pueden tener un carácter permanente o transitorio, por ejemplo, una reforma estructural, como las políticas sociales, en especial las nuevas relaciones laborales mencionadas en \cite{Bergara2017} debería tener un efecto permanente.

El modelo básico también permite ver como una economía en la cual los trabajadores enfrenten un riesgo mayor de perder los beneficios de un puesto ocupado, mayor $\lambda$, genera un mercado laboral menos eficiente. El caso extremo de $\lambda=0$, nos lleva a un punto de la curva que se situaría sobre el origen, un mercado sin desempleo ni vacantes. Si bien dicho caso es irrelevante en términos prácticos, muestra que sin la existencia del riesgo de perdida laboral, y sin trabajadores que transiten del empleo al desempleo ya que la \eqref{eq4} sería cero, estaríamos en una economía completamente eficiente. En otras palabras, la incertidumbre y los flujos laborales son relevantes para cuantificar la eficiencia de un mercado laboral.

Las modificaciones en la incertidumbre y flujos laborales, pueden deberse, por ejemplo, a las reformas estructurales sufridas por la economía uruguaya, que hayan afectado las condiciones laborales, por ejemplo, el aumento de poder de los sindicatos. Si bien $\lambda$ es una variable exógena en el modelo, podemos pensar dichos cambios como modificaciones en ella, siendo los mismos shocks transitorios o estructurales.

%\section{Estrategia Empírica}

%\subsection{Hipótesis}
%\subsection{SVAR parámetros variables y volatilidad estocástica}

La estrategia empírica a utilizar son los Vectores Autorregresivos estructurales con parámetros variables y volatilidad estocástica (TVP-VAR) siguiendo a \cite{Nakajima2011, Primiceri2005, Lubik2016b}, la cual puede considerarse una técnica novedosa \cite{Craven2014}. Esta metodología permite relajar el supuesto de una relación invariante entre vacantes laborales y desempleo, mediante la modelización de parámetros que siguen un proceso de markov de orden uno, a la vez que permite relajar el supuesto de una matriz de varianzas y covarianzas homogénea para todo el periodo. Como remarca \cite{Benati2013}, bien podría utilizarse test de quiebre estructural en vez de TVP-VAR. Sin embargo, \cite{Benati2007} muestra que los test de quiebres estructurales de \cite{BaiPerron1998, BaiPerron2003, Bai1997} ofrecen poca evidencia de quiebres cuando el proceso generador de datos (PGD) evoluciona como un paseo aleatorio, en contraposición a una metodología más flexible como \cite{Stock1996, Stock1998} que logra captar dicha evolución, por su parte \cite{Cogley2005} encuentran resultados similares. \cite{Benati2013} remarca que la utilización de TVP-VAR es robusta frente a la especificación de la variación temporal en los datos, mientras que los test de quiebres estructurales lo son solamente si el PGD tiene quiebres discretos.

Otra posible elección es la desarrollada por \cite{Barnichon2012, Hobijn2013}. Como sugieren \cite{Hobijn2013} el análisis no lineal es el método empírico más común en el análisis de la curva de Beveridge, sin embargo, no es el único, ellos utilizan una nueva forma basados en \cite{Barnichon2012} en el cual estiman el logaritmo del ratio de contrataciones sobre el stock de vacantes usando como regresores las contrataciones, separaciones, número de desempleados y empleados y el stock de vacantes. Desafortunadamente, no todas esas variables están disponibles en Uruguay para el periodo considerado. Descartadas estas dos metodologías alternativas, se sigue adelante con el TVP-VAR.


%\subsection{Identificación}

Para poder estimar un VAR estructural es necesario imponer restricciones de identificación sobre la matriz de varianzas y covarianzas para pasar de la forma reducida a la forma estructural. De esta forma, es posible descomponer el efecto de cada shock individual sobre las restantes variables endógenas del sistema, \cite{Hamilton1994}. Las parametrizaciones que se deseen imponer sobre la matriz de varianzas y covarianzas puede provenir o no de la teoría económica. En el primer caso suele suceder cuando una variable es publicada con rezago respecto de otra, o bien responden de forma diferente, por ejemplo una variable financiera y otra relacionada a bienes y servicios. En cualquier caso, las restricciones pueden ser de corto plazo, de largo plazo o de signo y los shocks pueden ser tanto permanentes como transitorios. 

%\section{Resultados}

%\section{Discusión}

%\section{Conclusiones}

 % Se carga el capítulo 03
  % \chapter{Presentación de los datos, Análisis, Discusión}

En algunas disciplinas, el capítulo Presentación de datos va acompañado del análisis o de la discusión de la información (\textit{Presentación y análisis de los datos}; \textit{Resultados y discusión}), en tanto que en otras, \textit{Presentación}, \textit{Análisis} y \textit{Discusión} son capítulos separados.
El objetivo de esta(s) parte(s) de la tesis es presentar los datos recabados y el análisis realizado a la luz de la bibliografía ya revisada. Se puede incluir la interpretación de los resultados (\textit{Discusión}) a partir del análisis de los datos, o también relacionarlos con estudios relevantes que se entienden pertinentes, aun si estos no se han consignado en los \textit{Fundamentos teóricos}, ya que se entiende que al analizar los datos pueden aparecer algunos que no se enmarcan teóricamente o que no se explican en el encuadre teórico o en estudios ya existentes.

Ahora a modo de ejemplo mencionamos el símbolo de los números reales utilizando el comando \verb|\gls{}| \gls{Real} y el comando \verb|\glssymbol{}| \glssymbol{Real}. Otro ejemplo es mencionar el tensor simétrico de tensiones \gls{sigma}, o un valor escalar  \gls{alph} o un conjunto vacío \gls{emptyset}.

\newpage 


\section{Título de sección}

Ejemplo de tabla

\begin{table}[h!]
\centering
\caption{Leyenda de tabla.}
\label{tab:comp}
\begin{tabular}{|c|c|c|}
  \hline
  $t$ (seg) & $x$(t) & $y$(t)\\
  \hline
  1 & 0.0000 & 0.0001\\
  2 & 0.5000 & 0.2498\\
  3 & 1.0000 & 1.0000\\
  4 & 1.5000 & 2.2403\\
  5 & 2.0000 & 4.0010\\
  6 & 2.5000 & 6.2459\\
  \hline
\end{tabular}
\end{table}

Ejemplo de figura.

\begin{figure}[h!]
\label{fig:comp}
\includegraphics[width=.8\textwidth]{imagenes/chap4/x_vs_y}
\caption{Leyenda de figura.}
\end{figure}
Ejemplo de ecuación:
\begin{equation}
y(x)=x^2
\end{equation}


%\subsection{Conceptos y Medición}

%\subsubsection{Desempleo} 

El desempleo se entiende como personas que buscan trabajo remunerado activamente pero no logran obtenerlo. Según el \cite{INE2019} ``se considera como desempleado a toda persona que durante el período de referencia considerado (última semana) no está trabajando por no tener empleo, que lo busca activamente y está disponible para comenzar a trabajar ahora mismo. Por definición, también son desocupados aquellas personas que no están buscando trabajo debido a que aguardan resultados de gestiones ya emprendidas y aquellas que comienzan a trabajar en los próximos 30 días''. 
Su cálculo se realiza en función de la tasa de desempleo calculada por el INE y series trabajadas por parte del IECON.

%\subsubsection{Vacantes}

En la construcción del índice se tiene como base al Conference Board Help Wanted OnLine (HWOL) de EEUU, debido a que se observa \cite{Barnichon2010, Shimer2005} que es un proxy bastante confiable respecto del Job Openings and Labor Turnover Survey (JOLTS, calculado por el Bureau of Labor Statistics). Según el \cite{JOLTS} una vacante disponible debe cumplir 1)Una posición especifica exista 2)El trabajador pueda comenzar en no más de 30 días 3)El empleador busca activamente fuera del establecimiento la vacante.\footnote{Traducción propia} 


%\subsection{Fuentes de datos}
Los datos utilizados provienen prácticamente de una única fuente de datos, la misma es el diario El País, sección avisos laborales, ``El Gallito''. A partir de ella, construyen \cite{Rama1988} quién calcula un índice de vacantes desde 1978 hasta 1987 con base 100 en 1980\footnote{Utiliza también dos fuentes de datos construidas por el MTSS. Combina las 3 en un índice sintético de vacantes laborales} y \cite{Urrestarazu1997} que obtiene un índice de vacantes desde 1987 hasta 1995. Es Urrestarazu quién, une ambas series estimando la cantidad de avisos en el trimestre octubre-diciembre de 1986 e igualándola al valor del índice en dicho trimestre, con ello obtiene un índice de vacantes laborales desde 1978 hasta 1995, sin embargo, en su trabajo solamente se presentan los datos desde 1981-3. Dado que ambos índices fueron expresados en función de la Población Económicamente Activa (PEA) las mismas pueden ser combinadas.

El Centro de Estudios de la Realidad Económica y Social (CERES) construyó un índice de vacantes laborales llamado ``Índice CERES de Demanda Laboral'' (ICDL) con frecuencia mensual y desestacionalizado, desde marzo de 1998 hasta 2014 con base agosto 1998. Si bien no ha sido posible encontrar una nota metodológica precisa y detallada, si se han encontrado publicaciones \cite{Ceres2012} que detallan fue construido relevando información de solicitudes de trabajo publicados en la sección de clasificados de la prensa uruguaya, pero sin detallar la misma. Sin embargo, dada la preponderancia de los avisos clasificados publicados en El País, sección ``El Gallito'' la posibilidad que los mismos no hayan formado parte del análisis es nula. Bien pueden haberse agregado otras fuentes de información, siendo clasificados de otros diarios.

Dada la falta de información respecto al ICDL, se buscó contrastar al mismo con otras fuentes de información. Para ello se utilizaron los datos generados por \cite{Alma2011}, donde la base de datos fue facilitada por los autores. Con ella se construyó un índice de frecuencia anual, el cual se comparó con el ICDL también anualizado. Lo que se observa es un comovimiento de los mismos y una correlación del orden del 97\%. También se analizaron ambas series en tasas de crecimiento, mediante una transformación logarítmica y posterior diferencia, nuevamente ambas mostraron resultados similares con una correlación en torno al 90\%.

En este trabajo se han conseguido datos de avisos laborales publicados en el diario El País, sección ``El Gallito'' para 3 periodos de tiempo, 1995-1998, agosto 1998 y 2013-2018. El método de obtención de los dos primeros fue el mismo. Los datos fueron recabados manualmente revisando de forma exhaustiva las secciones de las publicaciones semanales de ``El Gallito'' guardadas en la Biblioteca Nacional. En el tercer caso, se obtuvo la base de datos de ``El Gallito'', la misma fue solicitada formalmente al Diario El País el cual accedió a compartir los datos, bajo estricto uso académico.

Las secciones de El Gallito son 5: masculino, femenino, servicio doméstico, otros trabajos pedidos y avisos destacados. Cada sección tiene su particularidad, masculino y femenino son las secciones que registran sistemáticamente mayor cantidad de avisos en el orden de 300-500 semanales, servicio doméstico entre 50-100, mientras otros trabajados pedidos es marginal e insignificante registrando 1-5 avisos. De crucial importancia resulta avisos destacados, ya que, si bien la cantidad de avisos publicados suelen ser 100-200 hay multiplicidad de avisos con gran cantidad de puestos solicitados. El hecho de que solo se hayan registrado los puestos de avisos destacados es debido a una restricción de tiempo y al mayor detalle que ofrecían dichas publicaciones.

Las series principales son 3, aunque existen algunas auxiliares donde se contabilizan cuantos avisos publicaron más de 10, 20, 30, 40 y 50 puestos. Se construye una serie de avisos, otra de avisos filtrados y finalmente una de puestos. La serie de avisos incluye a todas las secciones, y simplemente refiere a la cantidad de avisos publicados, por lo tanto, dicha serie es idéntica a la obtenida por Urrestarazu para los años 1989-1995. La serie avisos filtrados, tiene un solo cambio y es que los avisos destacados son filtrados. Por último, la serie puestos trabaja con el supuesto que los avisos publicados en todas las secciones salvo en avisos destacados son la cantidad de puestos pedidos, mientras que para avisos destacados se analizó cada aviso individualmente, algunos fueron filtrados y de los restantes se contaron los puestos requeridos.

Los avisos filtrados tenían las siguientes características: ofrecían invertir en algún negocio, se pedían corredores los cuales no se consideran empleados, solicitaban vendedores independientes, exigían realizar un curso previo en algún instituto, eran ventas por catalogo o bijou a consignación, solicitaban distribuidores, pedían viajeros no exclusivos Estos filtrados se realizaron solamente para los dos primeros periodos.

El primer periodo es desde septiembre de 1995 hasta junio de 1998, con datos obtenidos revisando las publicaciones semanales de forma exhaustiva. De allí se obtiene información de cantidad de avisos laborales para la sección masculino, femenino, servicio domésticos, otros trabajos pedidos y avisos destacados. Además se obtiene la cantidad puestos laborales para avisos destacados. El número de avisos obtenidos son cerca de 110.000, mientras la cantidad de puestos laborales de avisos destacados son aproximadamente 32.000. 

El segundo son los avisos laborales publicados en agosto de 1998, la metodología fue la misma que en el caso previo. La importancia de dicho mes, es que el ICDL tiene base agosto 1998. Por lo cual, era indispensable conocer dicho valor, el mismo fue de 4300 avisos publicados.

El tercero desde 2013-2018, contiene toda la información publicada en los avisos laborales a excepción de los nombres de empresas catalogados como reservadas.

Los datos mencionados pueden verse en la figura \ref{fig:vacantes}. En el cuadrante (a) tenemos los datos recolectados por \cite{Alma2011}, los mismos vienen se dividen entre los avisos propiamente (es decir, cada publicación) y la cantidad de puestos totales en dichos avisos (puntos negros), ambos muestran un comovimiento. En el cuadrante (b) tenemos los datos utilizados en \cite{Urrestarazu1997} desde 1980 hasta 1995, los mismos representan la cantidad de avisos totales en cada año. En el cuadrante (c) tenemos las series de puestos y avisos recolectadas en este trabajo, donde se evidencia una correlación elevada de las series. En el recuadro (d) se observa la comparación entre las series de ICDL y IECON medidas en tasa de crecimiento anual, las mismas evidencian una correlación de 0.9. Finalmente en el rectángulo (e) tenemos el ICDL para el periodo de 1998 hasta 2014 con frecuencia mensual y desestacionalizado.

%\begin{figure}[h!]
%	\centering
%	\begin{subfigure}[b]{0.4\linewidth}
%		\includegraphics[width=\linewidth]{imagenes/chap4/iecon2000.png}
%		\caption{Avisos y Puestos IECON}
%	\end{subfigure}
%	\begin{subfigure}[b]{0.4\linewidth}
%		\includegraphics[width=\linewidth]{imagenes/chap4/VacantesUrrestarazu.png}
%		\caption{Vacantes Urrestarazu}
%	\end{subfigure}
%	\begin{subfigure}[b]{0.4\linewidth}
%		\includegraphics[width=\linewidth]{imagenes/chap4/MolinaAvisosyPuestos96.png}
%		\caption{Avisos recolectados 96-98}
%	\end{subfigure}
%	\begin{subfigure}[b]{0.4\linewidth}
%	\includegraphics[width=\linewidth]{imagenes/chap4/CeresIECON2000.png}
%	\caption{Tasas crecimiento CERES-IECON}
%\end{subfigure}
%	\begin{subfigure}[b]{0.5\linewidth}
%		\includegraphics[width=\linewidth]{imagenes/chap4/ICDL98.png}
%		\caption{ICDL 1998-2014}
%	\end{subfigure}
%	\caption{Análisis de los datos obtenidos}
%	\label{fig:vacantes}
%\end{figure}

La serie de vacantes en proceso de construcción no esta exenta de críticas, las cuales vale destacar. La primera es que los datos solo permiten realizar un análisis para Montevideo, sin embargo, se encuentran sistemáticamente avisos laborales de otros departamentos del interior\footnote{Uruguay se compone de 19 grandes divisiones geográficas denominadas ``departamentos''. Por un lado esta Montevideo, y por el otro el resto de los 18 departamentos del país, lo que se denomina ``interior''.}, en especial de Maldonado y Canelones. Para el periodo recabado de 1995 a 1998 no se identifica cuantos avisos son y no son de Montevideo, sin embargo, si se logra obtener dicho número para el periodo de 2013-2018. En dicho caso los avisos publicados de otros departamentos oscilan en torno al 5\% del total publicaciones laborales. 

En segundo lugar, existe un claro sesgo hacia puestos laborales de baja calificación, lo cual se logra observar para el periodo 2013-2018 en base al análisis conjunto de la figura 2 y 3, donde se observa claramente que la mayor cantidad de avisos corresponden a puestos de auxiliar, técnico/especialista, ejecutivo comercial y peón. Esto va en linea con el resultado obtenido por \cite{Alma2011} para el periodo 2000-2010 utilizando la misma fuente de datos, quienes realizar un análisis pormenorizado de la información publicada en cada aviso laboral. Por lo cual, se supone que dicho sesgo también existe para el periodo 1980-1996.

%\begin{figure}[h]
%	\includegraphics[width=\linewidth]{imagenes/chap4/newplot.png}
%	\caption{Cantidad de avisos laborales publicados en el diario El País, sección avisos laborales ``El Gallito'' entre 2013 y 2018. Los mismos son agrupados y ordenados de forma decreciente según el área en que se solicitan los avisos. El área, es una categoría propia del diario El País.}
%\end{figure}
%
%\begin{figure}[h]
%	\includegraphics[width=\linewidth]{imagenes/chap4/gallito_nivel_2013_2018.png}
%	\caption{Cantidad de avisos laborales publicados en el diario El País, sección avisos laborales ``El Gallito'' entre 2013 y 2018. Los mismos son agrupados y ordenados de forma decreciente según el nivel jerárquico en que se solicitan los avisos. El nivel jerárquico, es una categoría propia del diario El País.}
%\end{figure}

En tercer lugar, la existencia de informalidad es una característica propia de los países subdesarrollados a la cual Uruguay y en específico Montevideo no escapa. Ello se agrava para los años entre 1980 y 2005, debido a las graves crisis económicas de 1982 y 2001-2002. Claramente las vacantes laborales publicadas en la prensa no captan los puestos generados en la informalidad.

Por último, ``El Gallito'' ha sido al menos hasta 2000-2010 el lugar predilecto por las empresas para publicar sus avisos laborales. Es un hecho incuestionable, por lo cual tomarlo como una muestra representativa de avisos laborales formales para el departamento de Montevideo es razonable. Sin embargo, con la penetración de Internet y la creación de portales de anuncios laborales como ``buscojobs'', ``computrabajo'', ``neuvo'' y ``linkedin'', entre otros, la competencia en el mercado de publicaciones ha aumentando considerablemente, no quedando claro en la actualidad cual es la página que reúne la mayor cantidad de avisos. Por esta razón, la representatividad del gallito ha disminuido y ello debe ser tenido en cuenta en el análisis, al menos desde 2009-2010 en adelante, dada la inserción de ``buscojobs'', ``computrabajo'' en torno a los años 2006-2007.

%\subsection{Estadísticas Descriptivas}

%[Presentar estadísticas descriptivas de los datos. Gráficos de las series, y al ser series temporales análisis estadísticas básicas de las series tales como test de R.U, de Raíz estacional, desestacionalización,] % Se carga el capítulo 04
  % \include{capitulos/chap05} % Se carga el capítulo 05
 % Seguir copiando la linea de arriba para agregar más capítulos.
  \backmatter % Comando que generalos apéndices, anexos y bibliografía. NO COMENTAR
  
  % \bibliography{bibliografia/biblio_1,bibliografia/biblio_2,bibliografia/TesisBeveridgeCurve} % Agregar la cantidad de archivos .bib que se tengan para la bibliografía.
  % \bibend % No comentar
  % 
  \glosario 		         % Glosario, NO comentar
  % %
  \apenarabicnumbering
  \apenmatter				 % Apéndices, NO comentar
  % \input{apendice/apendice_A}
  % \input{apendice/apendice_B}
  % \input{apendice/apendice_C}
  % % Seguir copiando la linea de arriba para agregar más apéndices.
  % %
  \anexarabicnumbering
  \anexmatter				 % Anexos, NO comentar
%  \input{anexo/anexo_A}
  % Seguir copiando la linea de arriba para agregar más anexos.
  % 
  $body$
  
  
\end{document}

% ===== FIN DEL DOCUMENTO =====

