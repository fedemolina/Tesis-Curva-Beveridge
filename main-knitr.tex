\documentclass[msc,oneside,a4paper]{udelar}\usepackage[]{graphicx}\usepackage[]{color}
% maxwidth is the original width if it is less than linewidth
% otherwise use linewidth (to make sure the graphics do not exceed the margin)
\makeatletter
\def\maxwidth{ %
  \ifdim\Gin@nat@width>\linewidth
    \linewidth
  \else
    \Gin@nat@width
  \fi
}
\makeatother

\definecolor{fgcolor}{rgb}{0.345, 0.345, 0.345}
\newcommand{\hlnum}[1]{\textcolor[rgb]{0.686,0.059,0.569}{#1}}%
\newcommand{\hlstr}[1]{\textcolor[rgb]{0.192,0.494,0.8}{#1}}%
\newcommand{\hlcom}[1]{\textcolor[rgb]{0.678,0.584,0.686}{\textit{#1}}}%
\newcommand{\hlopt}[1]{\textcolor[rgb]{0,0,0}{#1}}%
\newcommand{\hlstd}[1]{\textcolor[rgb]{0.345,0.345,0.345}{#1}}%
\newcommand{\hlkwa}[1]{\textcolor[rgb]{0.161,0.373,0.58}{\textbf{#1}}}%
\newcommand{\hlkwb}[1]{\textcolor[rgb]{0.69,0.353,0.396}{#1}}%
\newcommand{\hlkwc}[1]{\textcolor[rgb]{0.333,0.667,0.333}{#1}}%
\newcommand{\hlkwd}[1]{\textcolor[rgb]{0.737,0.353,0.396}{\textbf{#1}}}%
\let\hlipl\hlkwb

\usepackage{framed}
\makeatletter
\newenvironment{kframe}{%
 \def\at@end@of@kframe{}%
 \ifinner\ifhmode%
  \def\at@end@of@kframe{\end{minipage}}%
  \begin{minipage}{\columnwidth}%
 \fi\fi%
 \def\FrameCommand##1{\hskip\@totalleftmargin \hskip-\fboxsep
 \colorbox{shadecolor}{##1}\hskip-\fboxsep
     % There is no \\@totalrightmargin, so:
     \hskip-\linewidth \hskip-\@totalleftmargin \hskip\columnwidth}%
 \MakeFramed {\advance\hsize-\width
   \@totalleftmargin\z@ \linewidth\hsize
   \@setminipage}}%
 {\par\unskip\endMakeFramed%
 \at@end@of@kframe}
\makeatother

\definecolor{shadecolor}{rgb}{.97, .97, .97}
\definecolor{messagecolor}{rgb}{0, 0, 0}
\definecolor{warningcolor}{rgb}{1, 0, 1}
\definecolor{errorcolor}{rgb}{1, 0, 0}
\newenvironment{knitrout}{}{} % an empty environment to be redefined in TeX

\usepackage{alltt}

\usepackage[
acronyms, 											 %utiliza el glosario de acronimos.
nohypertypes={acronym,notacion,simbolos,glosario},   %quita los links en el texto al glosario.
%nonumberlist,                                       %quita los links en los glosarios al texto.
nogroupskip,                                         %quita los espacios entre diferentes grupos dentro de un glosario.
nopostdot, 											 %quita el punto final en los acrónimos         .                           
]{glossaries}

\hypersetup{ colorlinks = true } % Hipervínculos: escribir "false" para imprimir o "true" para ver en digital.

% Ver los documentos de estilos bibliográficos para editar estas siguientes 2 líneas. Se deben de copiar a partir de los PDF de estilos bibliográficos y no es necesario que el estudiante las edite.
\usepackage{natbib} % Para algunos estilos bibliográficos
\bibliographystyle{estilos_bibliograficos/natbib/apalike}

\loadglossary


% \bibliography{bibliografia/TesisBeveridgeCurve.bib}
\IfFileExists{upquote.sty}{\usepackage{upquote}}{}
\begin{document}



    
    \title{Título del proyecto}
    \subtitle{Subtítutlo}
    \institutelogo{1} % Carga cantidad de logos seleccionados, con máximo de 3 logos.
    \author{Nombres integrantes del proyecto}{Apellidos integrantes del proyecto}
    \escritura{de} % Se indica que el programa de Posgrado sea "en" o "de" tal área.
    %
    \director{Prof.}{Nombre del Director de Tesis}{Apellido}{D.Sc.}
    \director{Prof.}{Nombre del Director de Tesis}{Apellido}{D.Sc.} % Comentar esta línea si se tiene solo un director de tesis.
    \codirector{Prof.}{Nombre del 1er Codirector}{Apellido}{D.Sc.}  % Comentar estas líneas si no son necesarias.
    \codirector{Prof.}{Nombre del 2do Codirector}{Apellido}{D.Sc.}
    \codirector{Prof.}{Nombre del 3er Codirector}{Apellido}{D.Sc.}
    \directoracademico{Prof.}{Nombre del Director Académico de Tesis}{Apellido}{D.Sc.}
    %
    \examiner{Prof.}{Nombre del 1er Examinador}{Apellido}{D.Sc.}
    \examiner{Prof.}{Nombre del 2do Examinador}{Apellido}{Ph.D.}
    \examiner{Prof.}{Nombre del 3er Examinador}{Apellido}{D.Sc.}
    \examiner{Prof.}{Nombre del 4to Examinador}{Apellido}{Ph.D.}
    \examiner{Prof.}{Nombre del 5to Examinador}{Apellido}{Ph.D.} % Comentar los que no son necesarios.
    %
    \graduatename{Fondo Sectorial de Datos}
    \institute{Facultad de Economía, Instituto de Estadística}{FCEA-IESTA}  % La primer institución es la principal.
    % \institute{Facultad de Medicina}{FMed}  % Agregar copias de esta línea para agregar instituciones.
    % \institute{Facultad de Agronomía}{FAgro}  % Agregar copias de esta línea para agregar instituciones. No se discrimina de que universidad es tal facultad.
    % \seconduniversity{Universidad XXXXX} % Se agrega el nombre de otra universidad. Comentar esta linea si no es necesaria.
    % \thirduniversity{Universidad YYYYY} % Se agrega el nombre de otra universidad. Comentar esta linea si no es necesaria.
    \graduatelocation{Montevideo}{Uruguay}
    %
    \date{31}{12}{2019} % Fecha del documento: día/mes/año
    % Palabras claves en español
    \keyword{1ra palabra clave}
    \keyword{2da palabra clave}
    \keyword{3ra palabra clave}
    \keyword{4ta palabra clave}
    \keyword{5ta palabra clave}
    % Palabras claves en inglés
    \foreignkeyword{1st keyword}
    \foreignkeyword{2nd keyword}
    \foreignkeyword{3rd keyword}
    \foreignkeyword{4th keyword}
    \foreignkeyword{5th keyword}
    % %     maketitle & frontmatter no funcionan.
    \maketitle
    % Comando que genera el título de la tesis.
    % \frontmatter
    % Comando que genera la portadilla, el catalogo y el tribunal de evaluación. NO COMENTAR
    % \SweaveInput{ded_agr_epi/dedicacion.Rnw}


% !Rnw root = main.Rnw
\chapter*{Agradecimientos}

Quisieramos agradecer a CEIBAL y ANII que han hecho posible este proyecto. En especial a Cecilia Marconi quien ha sido nuestra contraparte principal, a Jorge Briun quien hizo posible tener la aplicación web corriendo en el servidor del Instituto de Estadística y a todos quienes nos apoyaron.

    % \SweaveInput{ded_agr_epi/epigrafe.Rnw}


% !Rnw root = main.Rnw
\begin{abstract}

Escribir resúmen del proyecto

\end{abstract}




% !Rnw root = main.Rnw
\begin{foreignabstract}

In this work, we present ...

\end{foreignabstract}

    
    % \listadesimbolos
    % Lista de símbolos
    % \listadesiglas
    % Lista de siglas
    \tableofcontents
    % Tabla de contenidos. Compilar dos veces para ver los cambios completos.
    \mainmatter
    % Comando que genera las listas y capítulos. NO COMENTAR
    % Se incluyen los capítulos. Se pueden comentar los capítlos en los cuales no se está trabajando, para que el documento de trabajo sea más pequeño y compile más rápido.


% !Rnw root = main.Rnw
\chapter{Introducción}

El presente proyecto tiene por finalidad contribuir en la mejora de la calidad de la enseñanza a través de sistema de alerta temprana que permita predecir alumnos que estén en riesgo de rezago académico. El objetivo general del proyecto es desarrollar herramientas estadísticas que permitan evaluar y monitorear el uso de las plataformas educativas del Plan Ceibal en vínculo con los aprendizajes. Entender de qué forma el uso de las plataformas educativas afectan o influyen en el aprendizaje es crucial para mejorar su desarrollo e incentivar su uso. Así mismo desarrollar indicadores de uso de las plataformas que estén disponibles rápidamente es importante para el monitoreo de las mismas.
Adicionalmente, este proyecto es un antecedente para extrapolar dicha metodología a otras disciplinas, así como quedar a disposición de Plan Ceibal para introducir la misma en su plataforma de analíticas.

Plan Ceibal ha puesto a disposición de sus beneficiario desde 2013 dos plataformas educativas, la Plataforma Adaptativa de Matemática (PAM) y CREA2. CREA2 es un entorno virtual de aprendizaje que permite gestionar cursos, crear o compartir materiales didácticos y trabajar colaborativamente en grupos. Esta plataforma permite complementar las clases presenciales con la educación virtual. 

Por otro lado, Ceibal en Inglés es una iniciativa que surge en 2012 con el objetivo de apoyar la enseñanza de inglés. En primaria esta iniciativa intenta cubrir la falta de docentes mientras que en educación media se enfoca en la destrezas de oralidad mediante videoconferencias con un profesor remoto. Para la evaluación de aprendizaje de la lengua inglesa se han implementado evaluaciones anuales adaptativas en línea desde el 2014 para los niños de 4°, 5° y 6° año.

El uso de las computadoras de Ceibal y la introducción de las mismas en el proceso de enseñanza depende mucho de la propuesta pedagógica de los educadores. No todos los docentes han incorporado la herramienta en su trabajo en el aula. En este sentido es importante para evaluar el plan usar la información disponible sobre el uso de las
computadoras en el aula. Ceibal genera información a nivel de cada computadora individual que no está siendo explotado completamente, para monitorear el uso y aplicación del plan a niveles de baja agregación (grupos, escuelas, etc) de una forma sistemática. Hasta ahora el monitoreo sobre el uso de plataformas ha estado enfocado en la gestión y no en la generación de información a disposición del docente para su uso educativo. Por otro lado, se han realizado investigaciones particulares para desarrollar indicadores de uso en programas específicos (en convenio IESTA-CEIBAL). Ninguna de las iniciativas mencionadas son dinámicas, es decir no están disponibles con la
información actualizada cuando la misma es requerida.

Elaborar herramientas que permitan resumir la información, ponerla a disposición en tiempo real y aplicar métodos estadísticas para la evaluación y monitoreo son fundamentales para el éxito del plan y el diseño de estrategias educativas.

Este proyecto se centrará en la información generada por la plataforma CREA2 y su vínculo con el resultado de los aprendizajes de las pruebas adaptativas de inglés. Siendo posible extender los resultados para otros aprendizajes y plataformas. Para diseñar las herramientas descritas más adelante, se utilizarán los datos de uso de la plataforma CREA2 generadas durante el año 2015 y 2017 y los resultados de la prueba adaptativa de Ceibal en Inglés de 2017. En ambos casos, es posible utilizar una muestra anonimizada de los datos seleccionada en coordinación con los técnicos de CEIBAL. Si bien los datos son para un período concreto, la estructura y característica de los mismos son similares en todos los años. De esta forma las herramientas estadísticas son útiles para utilizar con información más reciente que el sistema tenga disponible.

Una primer pregunta que el presente proyecto pretende responder es cómo se relaciona el uso de la plataforma CREA2 con la performance en las pruebas adaptativas de inglés.

Marconi, Goyeneche y Cobo (2017) sugieren que el uso de CREA se correlaciona con los desempeños. En dicho trabajo, estudian esta relación comparando la distribución de performance en la prueba para distintos niveles del uso de la plataforma. Tomando como punto de partida este trabajo, se propone identificar de qué forma la información de uso de CREA2 puede ser utilizada para obtener nuevos indicadores de uso de la plataforma educativa. Los indicadores de uso en combinación con variables crudas, son los insumos para implementar buenos modelos predictivos para los resultados de las pruebas adaptativas de inglés. Será de interés también identificar umbrales de uso de la plataforma para los cuales se obtienen distintos niveles de aprendizaje de las pruebas adaptativas.

Como segundo objetivo este proyecto busca desarrollar indicadores para monitorear el uso de la plataforma a distintos niveles de análisis (clase, grado, escuela, departamento, etc) y a distintas ventanas temporales. Estos indicadores y resúmenes de información estarán implementados para diferentes actores, en particular está enfocado para el equipo de Ceibal en inglés, y en especial para los mentores que visitan las escuelas.. La evaluación del uso de la plataforma presenta debilidades que pueden ser mejoradas. El Plan cuenta con información diaria del uso de la plataforma a nivel individual que no está siendo utilizada para cumplir con este objetivo.

Para este segundo objetivo utilizaremos visualización estadística de datos que permitan resumir de forma sencilla para el usuario final patrones complejos en altas dimensiones. Para obtener visualización de alta calidad se utilizará ggplot2 en R ya que permite mayor flexibilidad que cualquier otro paquete en R y se respalda teóricamente en en la gramática de gráficos desarrollada por Wilkinson (1999).

La solución propuesta busca generar en tiempo real, de forma sencilla y sistematizada la información a través de un reporte dinámico en los distintos niveles de análisis que incluyan visualización estadística así como medidas de resumen e indicadores para el monitoreo y modelos predictivos de aprendizaje. Las herramientas serán generadas con
el software libre R combinando shiny y Rmarkdown para cumplir con este objetivo.
Aunque la implementación será utilizando datos del año 2015 y 2016 el proyecto pretende desarrollar herramientas suficientemente flexibles para aplicaciones sucesivas en otros años.

probando



% !Rnw root = main.Rnw
\chapter{Antecedentes}

En los últimos 30 años en Uruguay, a partir de la creación de distintos programas e iniciativas (PEDEClBA, ANII y CEIBAL) el país ha dado señales de su apuesta al desarrollo del sector de Ciencia e Innovación. Entendiendo que la generación de conocimientos y de capital humano son la base para el desarrollo del país.

En esa línea, se implementa en Uruguay como política pública de carácter universal el Plan Ceibal que forma parte de la iniciativa mundial One Laptop per Child (OLPC). Según lo establecido en la Ley No18.640 de creación del Centro Ceibal, uno de sus principales cometidos es “contribuir al ejercicio del derecho a la educación y a la inclusión social mediante acciones que permitan la igualdad de acceso al conocimiento y al desarrollo saludable de la infancia y de la adolescencia” (Marconi, 2017).

Plan Ceibal ha implementado el “modelo uno a uno” que consiste en otorgar un dispositivo (laptop o tablet) de su propiedad a cada alumno y docente de la enseñanza pública básica (Educación Inicial y Primaria, y Educación Media Básica). De esta forma ha logrado generar igualdad de acceso a la tecnología, así como se asegura el acceso a internet en todos los centros educativos públicos en sus 10 años de implementación en el país.

Una de las innovaciones más importantes en el sistema educativos de las últimas décadas a nivel mundial es la introducción de nuevas tecnologías que modifican la forma de enseñar y aprender conocimiento. En el sistema educativo el cambio tecnológico se plasma en cursos basados en la web, sistemas de gestión de contenidos para el aprendizaje, sistemas inteligentes y adaptativos de aprendizaje basados en la web entre otros. En este contexto el Plan Ceibal se empalma con el nuevo paradigma de aprendizaje con fuerte componente tecnológico.

Estas nuevas tecnologías en el aula no solamente presentan un desafío en cuanto a el paradigma de aprendizaje mediado por tecnología sino también un desafío para analizar los datos generados por el sistema. Los datos generados son datos en altas dimensiones, se cuentan con registros individuales de uso de las distintas tecnologías en el tiempo con variables a nivel individual contenidas en distintos sistemas de información. La estructura de los datos a su vez es compleja. Estas dos características hacen que los métodos estadísticos clásicos presenten dificultades por lo que es recomendable implementar técnicas estadísticas modernas que permitan descubrir patrones en datos con esta complejidad.

Los métodos de aprendizaje estadístico o machine learning consisten en el entrenamiento de un modelo de forma que aprenda diversos comportamientos usando informacion de un subconjunto de observaciones. Los métodos de aprendizaje automático son ampliamente utilizado en una variedad de problemas de aplicación en diversas disciplinas, tales como la economía, informática, biología, medicina, etc. La complejidad de los problemas de aplicación requieren de algoritmos automáticos que capturen las características fundamentales de los datos de forma de tener una adecuada performance predictiva para el problema de interés. El aprendizaje estadístico comprende un amplio conjunto de métodos que se pueden dividir en dos enfoques, aprendizaje supervisado y aprendizaje no supervisado. El aprendizaje supervisado implica que la variable de respuesta es conocida y el objetivo es construir un predicto automático para la misma usando información de las variables explicativas. Cuando la respuesta es discreta el problema es llamado de clasificación y cuando la respuesta toma valores continuos el problema es llamado de regresión.

En segundo lugar el aprendizaje no-supervisado implica que la variable de respuesta no es conocida, es decir no contamos con etiquetas para la misma. En este caso el problema se centra en el estudio de la estructura de los datos tratando de identificar grupos homegénos y la clasificación automática de los datos dentro de tales grupos, la posibilidad de identificar el mecanismo generador de los datos, por ejemplo la estimación de la densidad de una muestra. 

Los métodos que se utilizarán en este proyecto se enmarcan en el analisis supervisado, donde se encuentran una amplia variedad de métodos que han tenido gran éxito en sus aplicaciones. Algunos ejemplos son los métodos basados en árboles de decisión como CART (Breiman et al., 1984), utilizados tanto para la clasificación como para la regresión. Este método presenta un gran número de ventajas: presentación gráfica del modelo bajo la forma de un árbol binario, capacidad de manejar datos heterogéneos sin codificación, categóricos o continuos, el aporte de un índice de importancia para cada variable y el no ser afectado mayormente por la existencia de datos faltantes. Esta técnica ha sido objeto de numerosas extensiones (Loh, Wei (2008);Yin, (2014)). Sin embargo CART presenta el inconveniente mayor de ser inestable, es decir un pequeño cambio en la muestra de entrenamiento puede conducir a modelos predictivos totalmente distintos.

Los métodos de agregación de predictores aparecieron como solución al problema de inestabilidad de CART teniendio una gran importancia en el aprendizaje estadístico. Estos métodos combinan varios predictores estadísticos con la finalidad de obtener un predictor con major performance predictiva. Dentro de estas técnicas se puede citar a Bagging (Breiman, 1996), Boosting (Freund \& Schapire, 1997) y los Bosques Aleatorios (Random Forests, Breiman, 2001). Asimismo existen muchas extensiones a los métodos clásicos de agregación que varían tanto en los predictores a agregar así como la forma ne que son agregados en el modelo final.

Existen a su vez antecedentes específicos en el área de la educación ya que estas nuevas formas de aprender y enseñar basadas en tecnología requieren a su vez formas distintas de evaluar y monitorear las estrategias educativas. Recientemente, se ha identificado como “Educational Data Mining” a una sub-disciplina dedicada al estudio de las metodologías estadísticas para analizar los datos generados en sistemas educativos con el objetivo de mejorar la performance de dichos sistemas ( Baker, et.al., (2010) ). Thakar y Mehta (2015) presentan una revisión de artículos en este campo. Los objetivos de las investigaciones de este tipo son la creación de tipologías de estudiante (por ejemplo con técnicas de clustering) y más comúnmente la predicción de performance y/o de abandono en un curso o un programa de estudio. Los modelos predictivos pueden servir como indicadores tempranos que sirvan para detectar a estudiantes en riesgo. Rovira et. al. (2017) realizan una comparación de varios métodos de aprendizaje automático (Machine Learning) para predecir la nota final y el abandono en estudiantes universitarios.

Bakhshinategh, et.al., (2018) estudian tareas y aplicaciones existentes en el área de minería de datos educativos identificando 13 categorías de aplicaciones categorizadas según su objetivo (modelado de estudiantes, sistema de soporte para la toma de decisiones, sistema adaptativo, evaluación e investigación científica). Por su parte Baker, et.al., (2010), identifican 5 aproximaciones dentro de la minería de datos educativos: predicción, agrupamiento, minería de relaciones, descubrimiento con modelos y destilación de datos para el juicio humano. Estas dos referencias enmarcan nuestro proyecto dentro de los objetivos de modelado de estudiantes y soporte para la toma de decisiones. Dentro del primer objetivo la aproximación será basada en predicción y para el segundo enfocado en la destilación de datos para el juicio humano como lo especifica Baker, et.al., (2010).

Severin, E., Capota, C. (2011) mencionan que a pesar de la popularidad de la iniciativa de OLPC en Latin America y el Caribe, el monitoreo del impacto de las mismas no están debidamente cuantificado o cuentan con análisis parciales de la realidad. Mencionan a su vez la importancia fundamental de medir el impacto de la tecnología en el aprendizaje curricular. Las medidas de impacto pueden ser usadas para identificar estrategias pedagógicas y usar las tecnología que tiene mayor impacto en el aprendizaje.

En este contexto, es relevante el uso de técnicas estadísticas que permitan obtener mejor provecho de la enorme cantidad de información generada por CEIBAL para el monitoreo y el diseño de políticas educativas.

A nivel nacional, existen estudios que dan indicios sobre el impacto que tiene el
el uso de las plataformas educativas en el aprendizaje. En particular, la evaluación de impacto de la Plataforma Adaptativa de Matemática en los resultados de los aprendizajes realizado por el Centro de Investigaciones Económicas (CINVE) da cuenta de la ganancia de aprendizajes en matemática a partir de información longitudinal de una muestra de alumnos de la educación primaria. Esta primera evidencia a escala país, del impacto de una herramienta pedagógica de este tipo, muestra que la posibilidad de mejorar la calidad de la educación a través del uso de la tecnología es una alternativa real (Perera, M. et al, 2017).

Por otra parte, el estudio realizado sobre el efecto de la modalidad de enseñanza de inglés (virtual vs presencial) en el nivel de engagement de los alumnos con la plataforma CREA, y su asociación con los logros de los estudiantes dan cuenta de la importancia del rol docente, así como del contexto sociocultural, y del grado escolar en el mayor nivel de apropiación de la plataforma. Además se evidencia la asociación positiva entre mayores niveles de engagement y mejores performance por parte de los alumnos en la evaluación de aprendizajes (Marconi, C., et al. 2017)



% !Rnw root = main.Rnw
\chapter{Estrategia de Investigación}

Esto lo voy a dejar para el final.



% !Rnw root = main.Rnw
\chapter{Resultados}

El marco teórico del proyecto es la curva de Beveridge, que plantea una relación convexa hacia el origen entre vacantes laborales y desempleo, como puede observarse en la figura 4 del apéndice. Por tanto, los indicadores de los conceptos claves sobre los que se trabaja son el índice de vacantes laborales y la tasa de desempleo. Sin embargo, es posible enmarcar la curva de Beveridge bajo la teoría de búsqueda y emparejamiento. De esta forma, la misma se deriva en el modelo básico \cite{Pissarides2000} a partir de suponer la existencia de una función de matching, de suponer un proceso estocástico de Poisson mediante el cual se llenan las vacantes y que en un estado estacionario la tasa de variación de la tasa de desempleo debe ser nula. La función de matching, es un artilugio 

% <<>>=
% # Carga de paquetes y datos
% paquetes <- c("ggplot2", "data.table")
% sapply(paquetes, require, character.only = TRUE)
% sc <- readRDS(here::here("Datos", "Finales", "crea2-2017-seccion-cruzada.rds"))
% pn <- readRDS(here::here("Datos", "Finales", "crea2-2017-panel.rds"))
% @
% 
% <<>>=
% # Como presentar los datos.
% # 0ro. Presentación de los datos, ej, cantidad de alumnos, departamentos que participan, etc.
% # 1ro. Presentación general, básicamente usar lo que esta en la primera app.
% # 2do. Presentación de las pruebas de inglés, usar todo lo de la app cei.
% # 3ro. Presentación del modelo para predecir puntajes
% @
% 
% 
% <<label=tabla>>=
% # Presentación básica de los datos.
% # Tabla con departamentos, cantidad de escuelas, 
% kableExtra::kable(
%     sc[, .(Escuelas   = length(unique(id_centro)),
%            Alumnos    = length(unique(id)),
%            Engagement = round(median(ind_eng90), 2)
%            ),
%         keyby = .(Departamento = departamento)
%        ][, .(.SD[], Ranking = frank(-Engagement, ties.method = "first"))],
%     format = "latex"
% )
% @
% 
% En la tabla XXX tenemos a los 19 departamentos, más una fila que hace referencia a los alumnos sin departamento asignado, son los alumnos sin actividad en la plataforma CREA2. La segunda y tercera columna muestran respectivamente la cantidad de escuelas y alumnos que participan en la plataforma para departamento. Por último, Engagement y Ranking refieren a la mediana del índice de engagement y posición que ocupa cada departamento. Como es esperable, Montevideo es el departamento con mayor cantidad de escuelas y alumnos seguido por Canelones, mientras Flores es el departamento con menor cantidad de alumnos y escuelas, a la vez que el de mejor desempeño.
% 
% Esta última información puede ser vista claramente en el mapa de la figura XXX. Se observa un comportamiento diferente entre Montevideo y el Interior del país, mientras Montevideo muestra el peor desempeño, Flores destaca siendo el de mayor involucramiento en terminos medianos.
% <<>>=
% # Acá agregamos mapa del índice de engagement.
% @
% 
% Interesa ver la distribución en cada departamento, donde podamos observar el comportamiento de cada escuela y ver que tanta diferencia hay entre ellas. En el gráfico XXX observamos un diagrama de cajas por departamento, donde cada punto hace referencia a la mediana de una escuela. La caja representa el 50\% de los datos, desde 25\% (primer cuartil) hasta el 75\% (tercer cuartil), la linea interior es la mediana del departamento. Adicionalmente se agrega un gráfico de violin, lo que nos muestra la forma de la distribución del índice por departamento, mientras la linea horizontal es la mediana del índice a nivel nacional. Podemos observar que Montevideo se encuentra por debajo de la mediana nacional, al igual que Maldonado, Lavalleja, Cerro Largo y levemente Artigas y San José. Por el contrario, Flores y Colonia muestran valores cercanos al máximo del índice, Salto, Durazno y Río Negro se encuentran levemente sobre la mediana nacional mientras Florida, Rivera, Rocha, Soriano, Paysandú, Treinta y Tres y Tacuarembó obtienen un desempeño superior.
% 
% <<label=boxviolin, fig=TRUE>>=
% # Boxplot por departamento, más violin.
% # Primero el boxplot, luego el violin. Elegir el color y ancho de forma adecuada porque queda un tanto saturado.
% ggplot(sc[departamento != "sin_dato"], aes(x = departamento, y = ind_eng90, fill = departamento)) +
%     geom_boxplot(width=.6, color = "gray") +
%     geom_violin(trim = FALSE, width = 1.5, fill = "white") +
%     geom_abline(aes(intercept = median(ind_eng90), slope = 0)) +
%     viridis::scale_fill_viridis(discrete = TRUE) +
%     theme(axis.text.x = element_text(angle = 90), legend.position = "none")
% @
% 
% 
% <<>>=
% # A continuación analizamos las dimensiones del engagement a lo largo del año.
% @
% 
% 
% <<label= pruebasIngles, fig=TRUE>>=
% # Estadísticas descriptivas de las pruebas de inglés, listening y VRG.
% 
% # Alumnos por prueba de inglés en 2017, Listening.
% ggplot(sc, aes(x = nivel_listening17)) +
%             geom_bar(position = "stack", fill = "black", aes(y = (..count..)/sum(..count..))) +
%             labs(
%                 x = paste("Categorias"), 
%                 y = "Proporción de alumnos", 
%                 title = paste("Gráfico de barras")) +
%             theme(
%                 plot.title = element_text(hjust = 0.5, size = 14),
%                 plot.subtitle = element_text(hjust = 0.5),
%                 plot.caption = element_text(hjust = 0, face = "italic"),
%                 panel.grid.minor = element_blank(),
%                 panel.grid.major = element_blank(),
%                 panel.background = element_blank(),
%                 plot.background  = element_blank(),
%             )
% 
% # Alumnos por prueba de inglés en 2017, VRG.
% ggplot(sc, aes(x = nivel_vrg17)) +
%             geom_bar(position = "stack", fill = "black", aes(y = (..count..)/sum(..count..))) +
%             labs(
%                 x = paste("Categorias"), 
%                 y = "Proporción de alumnos", 
%                 title = paste("Gráfico de barras")) +
%             theme(
%                 plot.title = element_text(hjust = 0.5, size = 14),
%                 plot.subtitle = element_text(hjust = 0.5),
%                 plot.caption = element_text(hjust = 0, face = "italic"),
%                 panel.grid.minor = element_blank(),
%                 panel.grid.major = element_blank(),
%                 panel.background = element_blank(),
%                 plot.background  = element_blank(),
%             )
% @
% 
% 
% <<>>=
% # Presentar la diferencia en la distribución del índice cuando se consideran alumnos sin activdad.
% @
% 



% !Rnw root = main.Rnw
\chapter{Consideraciones finales}

probando


    % % Seguir copiando la linea de arriba para agregar más capítulos.
    % \input{PROOF/Child.Rnw}

% 
  \backmatter 
  % Comando que generalos apéndices, anexos y bibliografía. NO COMENTAR
  \bibliography{bibliografia/TesisBeveridgeCurve.bib}
  % Agregar la cantidad de archivos .bib que se tengan para la bibliografía.
  \bibend
  % No comentar
  \glosario
  % Glosario, NO comentar
  \apenarabicnumbering
  \apenmatter
  % Apéndices, NO comentar


% !Rnw root = Master.Rnw
\chapter{Datos procesados}




% !Rnw root = Master.Rnw
\chapter{Esto seguramente no va a ir}\label{Ape2}

%   % Seguir copiando la linea de arriba para agregar más apéndices.
%   %
  \anexarabicnumbering
  \anexmatter
  % Anexos, NO comentar


% !Rnw root = main.Rnw
\chapter{Tal vez lo borro}\label{Ane1}

XXXXX

  % Seguir copiando la linea de arriba para agregar más anexos.
  % 

\end{document}
