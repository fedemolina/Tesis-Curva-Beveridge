\chapter{Fundamentos teóricos}

Este capítulo incluye la revisión de la literatura, de los enfoques, teorías o conceptos pertinentes en que se fundamenta la investigación. Se basa fundamentalmente en la exposición de otros trabajos sobre el tema estudiado.

En diferentes tradiciones académicas este capítulo recibe distintas denominaciones: Marco teórico, Estado de la cuestión, Estado del arte. El objetivo de este capítulo es guiar al lector en la interpretación de trabajos que se han ocupado previamente de la cuestión central de la tesis u ofrecen herramientas analíticas o interpretativas. 

Algunas disciplinas incluyen aquí objetivos, hipótesis y justificación de la metodología, en tanto que otras exigen capítulos independientes para estos contenidos. Asimismo, algunos trabajos requieren un capítulo propio para los antecedentes de la investigación.



\section{XXXX}

Es usual que en \textit{Fundamentos teóricos} o en otras partes de la tesis el autor incluya vocabulario específico de la disciplina en un glosario.

Un glosario incluye una lista de términos y su explicación sucinta. El objetivo de este apartado es permitirle a un lector especializado en el área, aunque no necesariamente en la temática, comprender con mayor facilidad ciertos términos. 
Se organiza en forma alfabética y en el cuerpo de la obra se lo puede señalar con \textsc{versalita} la primera vez que se menciona, si este tipo de letra no fue utilizado con otro fin.

Se mencionan a modo de ejemplo tres posibles palabras a definir:
%
\begin{itemize}
\item \gls{adjetivo}
\item \gls{adjetivo_re}
\item \gls{adjetivo_cu}
\end{itemize}


Según \cite{Rodenburg2007} el uso de la curva de Beveridge ha pasado por periodos de amplia y baja utilización. Pese a ello, \cite{Elsby2015} muestra que ha logrado volverse un marco de análisis central en la macroeconomía, en especial la economía laboral. Ello se ha logrado fundamentalmente por los trabajos de \cite{Pissarides1985}, \cite{Mortensen1994} y \cite{Diamond1982} quienes dieron microfundamentos a la relación entre vacantes y desempleo de forma de alinear a la misma dentro de la economía neoclásica y, en especial al trabajo de \cite{Blanchard1989} que colocó la curva de Beveridge en la primera escena de la macroeonomía. La importancia de haber dado microfundamentos radica en que la misma pasa de ser un herramental descriptivo (como ha sido utilizada en Uruguay), a uno  de carácter explicativo e interpretativo de extrema utilidad para la política económica. Esto último no quiere decir que no halla sido utilizada previamente con dichos fines. 

Su aplicación en Uruguay puede remontarse a \cite{Rama1988}, \cite{DECON1993}, \cite{Urrestarazu1997} y \cite{Alma2011}, mientras que para un análisis de distintos países de la OCDE hasta 1990 puede verse \cite{Nickell2002} y hasta 2013 \cite{Hobijn2013}. En el caso de \cite{Rama1988}, su utilización fue con fines descriptivos para aproximarse a la descomposición de la desocupación en desempleo voluntario, de segmentación y desequilibrio. Su trabajo es el primero en Uruguay en plantear gráficamente la curva de Beveridge (1978-1988) y es quien construye el primer índice de vacantes conocido. Su objetivo se centra en cuantificar los componentes de la desocupación analizando la agregación de micro mercados de trabajo mediante un modelo de desequilibrio. Plantea que la curva de Beveridge puede utilizarse para cuantificar los componentes de segmenación y desequilibrio, no así el voluntario, debido a las variaciones de la PEA en la década de 1980.

\cite{Urrestarazu1997} construye sobre Rama, extiende el periodo de análisis hasta el año 1995 y, si bien su foco es al igual que Rama el análisis del desempleo de segmentación y los micro mercados mediante modelos de desequilibrio, estima por mínimos cuadrados ordinarios (MCO) una curva de Beveridge. Mediante una estimación restringida y otra no restringida (test Chow) pone a prueba la hipótesis de que exista un cambio estructural en 1990. Sus resultados concluyen que no se rechaza la hipótesis de existencia de un cambio estructural en la curva de Beveridge. Sin embargo, advierte siguiendo planteos formulados por Rama que las conclusiones obtenidas solo son válidas en la medida que la evolución del desempleo voluntario sea estable, característica común en países desarrollados, pero no así en Uruguay. 

Por su parte, \cite{DECON1993} estiman mediante una regresión lineal una curva de Beveridge entre 1980 y 1990 en la cual encuentran evidencia de una tendencia de la curva a desplazarse hacia afuera, indicando un mercado de trabajo con mayor rigidez, aunque se demarcan que la razón fundamental de ello sean salarios relativos inadecuados. 

Por último, \cite{Alma2011}, la utilizan en sus propias palabras ``con fines ilustrativos'' para mostrar una relación inversa y negativa para el periodo 2000-2009. En la misma se observa, lo que parece indicar un cambio de pendiente en la curva de Beveridge, sin embargo, como los mismos autores remarcan dada la escasez de datos no es posible identificar alteraciones relevantes.